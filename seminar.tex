%===============================================================================
% Zweck:    KTR-Seminar-Vorlage
% Erstellt: 16.10.2007
% Updated:  15.04.2013
% Autor:    U.K. / M.G.
%===============================================================================

%===============================================================================
% Zum Kompilieren pdflatex und bibtex ausführen.
%	Konfiguration: in texmaker unter Benutzer -> Benutzerbefehle einen neuen Befehl anlegen: pdflatex -synctex=1 -interaction=nonstopmode %.tex | bibtex % | makeindex %.nlo -s nomencl.ist -o %.nls | pdflatex -synctex=1 -interaction=nonstopmode %.tex | pdflatex -synctex=1 -interaction=nonstopmode %.tex
%	Entsprechende Informationen in den config/metainfo verändern
% Zur Auswahl der Sprache im folgenden Befehl
% ngerman für deutsch eintragen, english für Englisch.
%===============================================================================

% Options ngerman, english
\newcommand{\lang}{ngerman}

\documentclass[pdftex, journal, onecolumn, a4paper, 12pt, \lang]{IEEEtran}
\usepackage[]{algorithm2e}
%===============================================================================
% zentrale Layout-Angaben und Befehle
%===============================================================================
% What You should change:
% Here goes your name
\author{Author}
% and the title of your seminar
\newcommand{\subtitle}{Your Topic on the Seminar}
% the date of the submission
\date{\today}

% What is already done

\newlanguagecommand{\semester}
\addtolanguagecommand{\semester}{english}{Summer Term 2020}
\addtolanguagecommand{\semester}{ngerman}{Sommersemester 2020}

\newlanguagecommand{\ltitle}
\addtolanguagecommand{\ltitle}{english}{Fog Computing in Next Generation Networks}
\addtolanguagecommand{\ltitle}{ngerman}{Fog Computing in Next Generation Networks}

\title{\ltitle}
\newcommand{\supervisor}{Prof. Dr. Udo Krieger}

\gittrue
\seminartrue

\input{config/layout}
%===============================================================================
% LATEX-Dokument
%===============================================================================

\input{config/hyphenation}
\begin{document}

\maketitle

\pagenumbering{Roman}
\setcounter{page}{2}
%
\tableofcontents
% Einstellungen f\"{u}r Literaturverzeichnis
\newpage
\addcontentsline{toc}{section}{\listfigurename}
\listoffigures
\newpage
\addcontentsline{toc}{section}{\listtablename}
\listoftables
\newpage
\printnomenclature
%===============================================================================
% LATEX-Dokument: Kapitel laden
%===============================================================================
%
\newpage
\pagenumbering{arabic}
\setcounter{page}{1}


%
% hier einzelne Kapitel mit \input{Kapitel-File} einf\"{u}gen
%
\input{chapters/exampleContent}
%
%===============================================================================
% LATEX-Dokument: Literaturverzeichnis
%===============================================================================
%
\newpage
\phantomsection
% Einstellungen f\"{u}r Literaturverzeichnis
\addcontentsline{toc}{section}{\bibname}

\bibliographystyle{IEEEtran}
% argument is your BibTeX string definitions and bibliography database(s)
\bibliography{literature/bib}
% Nutzung von Bibtex:
% hier den bib-file einbinden
%
% GATHER{bibfile.bib}
% \footnotesize
% \bibliography{bibfile}
% ansonsten: bbl als tex Datei einbinden
 %\input{KTR-Seminar-Literatur.tex}
%===============================================================================
% LATEX-Dokument: Literaturverzeichnis
%===============================================================================
%
\end{document}
